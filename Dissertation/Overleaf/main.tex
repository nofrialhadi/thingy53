% ----------------------------------------------------------------------
%  UNIVERSITY OF LIMERICK

% Jun 2024
% Updated by Nofri Alhadi - Jun 2024
% This template is updated for the Master's Degree Dissertation

% 2023
% Template for PhD Thesis by Publication
% Prepared by Hafiz Ahmad Awais Chaudhary
% https://www.linkedin.com/in/ahmad91/

% Department of Computer Science and Information Systems - University of Limerick, Ireland
% Based on Giuseppe Torre, Harish Bhandari's and Jakob Suckale templates  

% ----------------------------------------------------------------------

% ----------------------------------------------------------------------
%                  Import Package and Library
% ----------------------------------------------------------------------

%: Style file for Latex
% Most style definitions are in the external file PhDthesisPSnPDF.
% In this template package, it can be found in ./Latex/Classes/


\documentclass[openright,12pt]{Latex/Classes/PhDthesisPSnPDF}
%%%%%openany book twoside
%\usepackage{fp}     
%\usepackage{biblatex}  % temporary
\usepackage[nomessages]{fp}
\usepackage{natbib}
\usepackage{url}
\usepackage{alltt} 
\usepackage{multirow}
\usepackage{color}
\usepackage{graphicx}
\usepackage{subfig}
\usepackage{pdfpages}
%\usepackage{float}
\usepackage{array}
\usepackage{tcolorbox}
\usepackage[acronym,toc]{glossaries}
\usepackage{pgfgantt}
\usepackage[utf8]{inputenc}
\usepackage{url}
\usepackage{tikz}
\usetikzlibrary{backgrounds,calc,calendar}
\usepackage{pgfgantt}
\usepackage{pgfcalendar}
\usepackage{datetime2}
\usepackage{xcolor}
\usepackage{etoolbox}
\usepackage{calc}
\usepackage{ifthen}
\usepackage{fmtcount}      
\usepackage{pgffor}        
\usepackage{pdflscape}     
\usepackage{todonotes}     
\usepackage{xcolor}       
\usepackage{caption}
\usepackage{geometry}
\usepackage{pythonhighlight}
\usepackage{comment}
\usepackage{nicematrix}
%\usepackage{underscore}
%\usepackage[english]{babel}
\newcommand*{\escape}[1]{\texttt{\textbackslash#1}}
\usepackage{listings}
\lstset{ 
  backgroundcolor=\color{mygray},   % choose the background color; you must add \usepackage{color} or \usepackage{xcolor}; should come as last argument
  upquote=true,
  columns=fullflexible,
  basicstyle=\ttfamily\small, 
  literate={*}{{\char42}}1
         {-}{{\char45}}1
         {\ }{{\copyablespace}}1,% the size of the fonts that are used for the code
  breakatwhitespace= false,         % sets if automatic breaks should only happen at whitespace
  breaklines=true,                 % sets automatic line breaking
  captionpos=b,                    % sets the caption-position to bottom
  commentstyle=\color{black},    % comment style
  deletekeywords={...},            % if you want to delete keywords from the given language
  escapeinside={\%*}{*)},          % if you want to add LaTeX within your code
  extendedchars=true,              % lets you use non-ASCII characters; for 8-bits encodings only, does not work with UTF-8
  firstnumber=1000,                % start line enumeration with line 1000
  frame = shadowbox,	                   % adds a frame around the code
  keepspaces=true,                 % keeps spaces in text, useful for keeping indentation of code (possibly needs columns=flexible)
  keywordstyle=\color{black},       % keyword style
  language= C,                 % the language of the code
  morekeywords={*,...},            % if you want to add more keywords to the set
  rulecolor=\color{black},         % if not set, the frame-color may be changed on line-breaks within not-black text (e.g. comments (green here))
  showspaces=false,                % show spaces everywhere adding particular underscores; it overrides 'showstringspaces'
  showstringspaces=false,          % underline spaces within strings only
  showtabs=false,                  % show tabs within strings adding particular underscores
  showlines = true,
  stepnumber=2,                    % the step between two line-numbers. If it's 1, each line will be numbered
  stringstyle=\color{darkred},     % string literal style
  tabsize=4,	                   % sets default tabsize to 2 spaces
  title=\lstname                   % show the filename of files included with \lstinputlisting; also try caption instead of title
}

 
\def\UrlBreaks{\do\/\do-}

\setcitestyle{square}

%\makeglossaries

\newcommand{\urlwofont}[1]
    {\urlstyle{same}\url{#1}}

% to display grey publication box
\definecolor{lightgrey}{rgb}{0.9, 0.9, 0.9}

% to display Math
\usepackage{mathtools}

% to Display C++ code
\definecolor{c++red}{rgb}{0.6,0,0} % for strings
\definecolor{c++green}{rgb}{0.25,0.5,0.35} % comments
\definecolor{c++purple}{rgb}{0.5,0,0.35} % keyword
\usepackage{listings}
\lstset{language=C++}

%% Define a new 'Leo' style for the package using a smaller font.
\makeatletter  
\def\url@leostyle{%
  \@ifundefined{selectfont}{\def\UrlFont{\sf}}{\def\UrlFont{\small\ttfamily}}}
\makeatother
%% Now, actually use the newly defined style.
\urlstyle{leo}


% ----------------------------------------------------------------------


% ----------------------------------------------------------------------
%                  TITLE PAGE: name, degree,..t
% ----------------------------------------------------------------------
% below is to generate the title page with crest and author name

% If output to PDF then put the following in the PDF header
\ifpdf  
    \pdfinfo { /Title  (PhD and MPhil Thesis Classes)
               /Creator (TeX)
               /Producer (pdfTeX)
               /Author (Name Surname email@email.com)
               /CreationDate (D:YYYYMMDDhhmmss)  %format D:YYYYMMDDhhmmss
               /ModDate (D:YYYYMMDDhhmm)
               /Subject (Main subjects of your thesis)
               /Keywords (put here few keywords relevant to your PhD work) }
    \pdfcatalog { /PageMode (/UseOutlines)
                  /OpenAction (fitbh)  }
\fi

% define Title
\title{Optimizing Smart Agriculture for Chicken Egg Hatch Environments Using the Internal Sensors of the Nordic Semiconductor Thingy:53}

\supervisornamefirst{\href{mailto:tiziana.margaria@ul.ie}{\textbf{Prof. Dr. Tiziana Margaria}}}

\supervisornamesecond{\href{mailto:amandeep.singh@ul.ie}{\textbf{Amandeep Singh}}}

\submittedtext{Submitted to the University of Limerick \\ in partial fulfilment of the requirements for the degree of}


% The section below defines www links/email for authors and institutions
% They will appear on the title page of the PDF and can be clicked

%\ifpdf
    \author{\href{mailto:nofrialhadi@gmail.com}{\textbf{Nofri Alhadi - 23188901}}}
  
    %  \cityofbirth{born in XYZ} % uncomment this if your university requires this
    %  % If the city of birth is required, also uncomment 2 sections in PhDthesisPSnPDF
    %  % Search for the "city", and you'll find them.

    % Here are the links to the Institution you belong to
    \researchcentre{\href{https://lab.ie/}{lab name}}
    
    \department{\href{https://www.ul.ie/scieng/schools-and-departments/department-computer-science-and-information-systems}{Department of Computer Science and Information Systems}}
    
    \faculty{\href{https://www.ul.ie/scieng}{Faculty of Science and Engineering}}
    
    \university{\href{https://www.ul.ie/}{University of Limerick}}

    % The crest is a graphics file of the logo of your research institution.
    % Place it in ./00_frontmatter/figures and specify the width
    \crest{\includegraphics[scale=0.45]{figures/UL_logo_new.jpg}}
  
    % If you are not creating a PDF, then use the following. The default is PDF.
%\else
    % \author{YourName}
    % \cityofbirth{born in XYZ}
    % \collegeordept{CollegeOrDept}
    %\university{University}
    %\crest{\includegraphics[width=4cm]{logo.png}}
%\fi

%\renewcommand{\submittedtext}{change the default text here if needed}

\degree{Master of Science in Artificial Intelligence and Machine Learning}
\degreedate{August 2024}


% ----------------------------------------------------------------------
       
% turn of those nasty overfull and underfull hboxes
\hbadness=10000
\hfuzz=50pt

\begin{document}

%: Macro file for Latex
% Macros help you summarise frequently repeated Latex commands.
% Here, they are placed in an external file /Latex/Macros/MacroFile1.tex
% A macro that you may use frequently is the figure macro (see introduction.tex)
\include{Latex/Macros/MacroFile1}
%\language{english}

% sets line spacing
\renewcommand\baselinestretch{1.2}
\baselineskip=18pt plus1pt


% ----------------------- generate cover page ------------------------

\maketitle  % command to print the title page with the above variables
%\newpage

% ----------------------- cover page back side ------------------------
% Your research institution may require reviewer names, etc.
% This cover back side is required by 

\begin{comment}
\begin{table}
    \vspace*{-1cm}
    \hspace{0.5cm}
    \renewcommand{\arraystretch}{1}
    
    \begin{tabular}{p{2.2cm} p{10.5cm} }
            \textbf{Supervisor:} & Prof. Dr. Tiziana Margaria   \\
             & Chair of    \\
             & Department of Computer Science and Information Systems  \\
             & University \emph{of} Limerick  \\
             & Ireland \\
             &  \\
             
            \multirow{2}{2.2cm}{\textbf{Internal Examiner:}} & Dr. ...  \\
             & Department of Computer Science and Information Systems  \\
             & University \emph{of} Limerick  \\
             & Ireland \\
             &  \\
             
            \multirow{2}{2.2cm}{\textbf{External Examiner:}} & Prof. Dr....  \\
             & Division of Software ...s  \\
              & University  \\
              & Country  \\
             &  \\
            
            \textbf{Chair:} & Dr. ...  \\
             & Department of Computer Science and Information Systems  \\
             & University \emph{of} Limerick  \\
             & Ireland \\
             &  \\
             
            \multicolumn{2}{@{} l}{\textbf{\hspace{0.1cm} Day of the defence: ..... 2023}}\\
            &  \\
            &  \\
            &  \\
            &  \\
             
            \multicolumn{2}{@{} r}{\textbf{Signature from the head of PhD committee:}}\\
    \end{tabular}
\end{table}
\end{comment}

% --------------------------------------------------------------
%                  FRONTMATTER: dedications, abstract,...
% --------------------------------------------------------------

%\cleardoublepage 
% This file is called up by main.tex
% content in this file will be fed into the main document
% Thesis statement of originality -------------------------------------
% Depending on the regulations of your faculty, you may need a declaration like the one below. This specific one is from the medical faculty of the University of Dresden.

\begin{declaration}        % This creates the heading for the declaration page

I hereby declare that I have written this thesis without the prohibited assistance of third parties and without utilizing any aids other than those specified. Any ideas or information obtained directly or indirectly from external sources have been duly acknowledged. This thesis has not been previously submitted in an identical or similar form to any Irish or foreign examination board. \par \vspace*{1ex}

The thesis work was carried out in 2024 under the supervision of Prof. Dr. Tiziana Margaria and Amandeep Singh at the University of Limerick. \par \vspace*{1ex}


\vspace{50pt} % Vertical whitespace


\noindent\rule{150pt}{0.1pt}    \par
\textbf{Nofri Alhadi}   \\
Limerick, 2024  \par 



\end{declaration}


% ----------------------------------------------------------------------
% This file is called up by main.tex
% content in this file will be fed into the main document
% Thesis acknowledgements -----------------------------------------------------

\begin{acknowledgements} % This creates the heading for the acknowledgements

\noindent   I would like to express my deepest gratitude to everyone who has supported me throughout this project. First and foremost, I would like to thank my supervisors, Prof. Dr. Tiziana Margaria and Amandeep Singh, for their invaluable guidance, encouragement, and expertise, which have been instrumental in shaping this work. Your patience and insightful feedback have genuinely enriched this study. \par \vspace*{1ex}

I would also like to extend my heartfelt thanks to my wife, Rizki Amalia, for her unwavering support, understanding, and love throughout this journey. Her encouragement and belief in me have been a constant source of strength, especially during the most challenging moments of this project. \par \vspace*{1ex}

Additionally, I would like to acknowledge my colleagues and peers in the Master of Science in Artificial Intelligence and Machine Learning 2023 for their constructive discussions and feedback. \par \vspace*{1ex}

Lastly, I want to recognize the Department of Computer Science and Information Systems, Faculty of Science and Engineering, University of Limerick, for providing the necessary resources and support that made this project possible. \par \vspace*{1ex}

Thank you all for your significant contributions to this work.




\end{acknowledgements}
%\end{acknowledgmentslong}

% ------------------------------------------------------------------------
% This file is called up by main.tex
% content in this file will be fed into the main document
% Thesis dedication -----------------------------------------------------

\begin{dedication}        % This creates the heading for the dedication page

    To my wife, Rizki Amalia, whose endless love, support, and encouragement have been my constant strength and inspiration throughout this journey. And to my supervisors, Prof. Dr. Tiziana Margaria and Amandeep Singh, whose invaluable guidance, wisdom, and patience have been instrumental in completing this work. This thesis is dedicated to all of you with heartfelt gratitude. \par



\end{dedication}

% ---------------------------------------------------------------------- 

% This file is called up by main.tex
% content in this file will be fed into the main document
% Thesis Abstract -----------------------------------------------------

%\begin{abstractslong}    %uncommenting this line gives a different abstract heading
\begin{abstracts}        % This creates the heading for the abstract page

The successful incubation of chicken eggs requires precise control of environmental conditions throughout the entire process. This proposal for a thesis aims to investigate how the Nordic Semiconductor Thingy:53, an IoT device equipped with a range of internal sensors, can be utilized to optimize and automate the management of egg-hatch environments. The research will utilize Thingy:53’s BME688 sensor to monitor temperature, humidity, pressure, and gas (air quality) and the BH1749 light sensor to detect light levels. The goal is to develop a reliable real-time monitoring and control system by reducing human intervention through automated data collection, real-time alerts, and comprehensive data analytics, ultimately enhancing hatch rates and operational efficiency. Machine learning algorithms will predict optimal hatching conditions based on historical data, enabling proactive adjustments to the incubation environment. Integrating with cloud platforms will allow long-term data storage and advanced analytics, providing valuable insights for ongoing improvement. This study will assess the effectiveness of Thingy:53 in maintaining optimal hatching conditions and its potential as a cost-effective, sustainable tool in poultry operations. The results will contribute to the advancement of smart agricultural practices and lay the groundwork for future innovations in the industry.

\end{abstracts}
%\end{abstractlongs}
% ---------------------------------------------------------------------- 
%% This file is called up by main.tex
% content in this file will be fed into the main document
% Thesis glossary -----------------------------------------------------

%\begin{glossary} % This creates the heading for the acknowledgements

\newacronym{DSL}{DSL}{Domain-Specific Language}
\newacronym{CPS}{CPS}{Cyber-Physical Systems}






%\end{glossary}

%\printglossaries

%: ----------------------- table of contents ------------------------

\setcounter{secnumdepth}{2} % organisational level that receives a numbers
\setcounter{tocdepth}{3}    % print table of contents for level 3
\tableofcontents            % print the table of contents
% levels are: 0 - chapter, 1 - section, 2 - subsection, 3 - subsection


% ----------------------- list of figures/tables ------------------------

\listoftables  % print list of tables

% ----------------------- list of figures ------------------------

\listoffigures	% print list of figures

% ----------------------- glossary ------------------------

% Tie in external source file for definitions: /00_frontmatter/glossary.tex
% Glossary entries can also be defined in the main text. See glossary.tex
%% This file is called up by main.tex
% content in this file will be fed into the main document
% Thesis glossary -----------------------------------------------------

%\begin{glossary} % This creates the heading for the acknowledgements

\newacronym{DSL}{DSL}{Domain-Specific Language}
\newacronym{CPS}{CPS}{Cyber-Physical Systems}






%\end{glossary}

%\printglossaries 
%
%\begin{multicols}{2} % \begin{multicols}{#columns}[header text][space]
%\begin{footnotesize} % scriptsize(7) < footnotesize(8) < small (9) < normal (10)
%
%\printnomenclature[1.5cm] % [] = distance between entry and description
%\label{nom} % target name for links to glossary
%
%\end{footnotesize}
%\end{multicols}

%\printglossary[type=\acronymtype]
%\printglossary[type=\acronymtype,nonumberlist, toctitle=List of Abbreviations, title=List of Abbreviations]
%\printglossary[type=\acronymtype,nonumberlist]


% --------------------------------------------------------------
%                  MAIN DOCUMENT SECTION
% --------------------------------------------------------------

% The main text starts here with the introduction, 1st chapter,...
\mainmatter

% ----------------------- Chapters ------------------------
\renewcommand{\chaptername}{} % uncomment to print only "1" not "Chapter 1"

\include{01_introduction/introduction}                      % Introduction
\include{02_related_work/related_work}                      % Related Work     
\include{03_design_process/design_process}                  % Design Process and Timeline
\include{04_system_implementation/system_implementation}    % System Implementation
\include{05_data_analytic/data_analytic}                    % Data Analytic Workflows
\include{06_results/results}                                % Results and Discussion
\include{07_conclusions/conclusions}                        % Conclusion and Future work

% ----------------------- bibliography ------------------------

\Urlmuskip=0mu plus 1mu
\bibliography{references/references}
\bibliographystyle{Latex/IEEEtran_bib}
%\bibliographystyle{plainnat}
%\bibliographystyle{IEEEtran}

%plainnat: Basic style with author-year.
%abbrvnat: Abbreviated first names.
%unsrtnat: Unsorted, references appear in the order they are cited.
%apalike: APA-style references.

% Various bibliography styles exit. Replace above style as desired.
%\begin{small} % tiny(5) < scriptsize(7) < footnotesize(8) < small (9)
%\bibliographystyle{Latex/IEEEtran_bib}
%\bibliographystyle{Latex/splncs04}
%\renewcommand{\bibname}{References} % changes the header; default: Bibliography
%\bibliography{References-Bibtex} % adjust this to fit your BibTex file
%\end{small}
%\end{multicols}
% in-text refs: (1) (1; 2)
% ref list: alphabetical; author(s) in small caps; initials last name; page(s)
%\bibliographystyle{Latex/Classes/PhDbiblio-case} % title forced lower case
%\bibliographystyle{Latex/Classes/PhDbiblio-bold} % title as in bibtex but bold
%\bibliographystyle{Latex/Classes/PhDbiblio-url} % bold + www link if provided
%\bibliographystyle{Latex/Classes/jmb} % calls style file jmb.bst
% in-text refs: author (year) without brackets
% ref list: alphabetical; author(s) in normal font; last name, initials; page(s)
%\bibliographystyle{plainnat} % calls style file plainnat.bst
% in-text refs: author (year) without brackets
% (this works with package natbib)
%\chapter*{References}
%\addcontentsline{toc}{chapter}{References}


% --------------------------------------------------------------
%:                  BACK MATTER: appendices, refs,..
% --------------------------------------------------------------

\include{99_backmatter/appendixes} 

% -------------------------------------------------------------


\end{document}
